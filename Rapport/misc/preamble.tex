\documentclass[svgnames,a4paper,11pt,fleqn,openright]{report}
\usepackage{tikz}
\usepackage{verbatim}
\usepackage{nameref}
\usepackage{parskip}
\definecolor{light-gray}{gray}{0.8}
\usepackage{fourier}
\usepackage{array}
\usepackage[compact, explicit]{titlesec}
\usepackage{pdfpages}
\usepackage{colortbl}
\usepackage{url}
\usepackage[utf8]{inputenc}
\usepackage{listings}
\usepackage{amsfonts}
\usepackage{amsmath}
\usepackage{amssymb}
\usepackage{longtable} 
\usepackage{pdflscape}
\usepackage{makecell} %diagonal streg i celler, yay!
\usepackage[]{todonotes} %organized notes, yay! skriv "disable" i "[]" for at skjule notes
%\usepackage{showframe}
\marginparwidth = 105pt
\usepackage{wrapfig}

\usepackage[backend=bibtex,sorting=none]{biblatex}
\addbibresource{misc/litteratur.bib}

\newcommand*\chapterlabel{}
\titleformat{\chapter}
  {\gdef\chapterlabel{}
   \normalfont\sffamily\Huge\bfseries\scshape}
  {\gdef\chapterlabel{\thechapter.\ }}{0pt}
  {\begin{tikzpicture}[remember picture,overlay]
    \node[yshift=-2cm] at (current page.north west)
      {\begin{tikzpicture}[remember picture, overlay]
        \draw[fill=light-gray] (0,0) rectangle
          (\paperwidth,2cm);
        \node[anchor=east,xshift=.9\paperwidth,rectangle,
              rounded corners=20pt,inner sep=11pt,
              fill=Gray]
              {\color{white}\chapterlabel#1};
       \end{tikzpicture}
      };
   \end{tikzpicture}
  }
\titlespacing*{\chapter}{0pt}{10pt}{-80pt}
\titlespacing*{\section}{0px}{20pt}{-10px}
\titlespacing*{\subsection}{0px}{20pt}{0px}
\titleformat{\section}{\large\bfseries}{\thesection}{1em}{#1\hrule}

\lstset{
backgroundcolor=\color[rgb]{0.95, 0.95, 0.95},
tabsize=2,
rulecolor=,
basicstyle=\scriptsize,
upquote=true,
aboveskip={1.5\baselineskip},
columns=fixed,
showstringspaces=false,
numbers=left,
extendedchars=true,
breaklines=true,
prebreak = \raisebox{0ex}[0ex][0ex]{\ensuremath{\hookleftarrow}},
frame=single,
showtabs=false,
showspaces=false,
showstringspaces=false,
captionpos=b,
identifierstyle=\ttfamily,
language=C,
keywords=[1]{function, if, else, or, and, for, while, foreach, from, to, return, uint16_t, PacketType, Packet, char},
keywords=[2]{OUTPUT,HIGH,LOW,ON,OFF, true, false},
keywords=[3]{derp, delay, digitalWrite, pinMode, digitalRead, analogWrite, analogRead, write, read, wait, print, void, int, break, wait}
keywordstyle=\color[rgb]{1.0,0,0},
keywordstyle=[1]\color[rgb]{0,0,0.75},
keywordstyle=[2]\color[rgb]{0.5,0.0,0.0},
keywordstyle=[3]\color[rgb]{0.127,0.427,0.514},
keywordstyle=[4]\color[rgb]{0.4,0.4,0.4},
commentstyle=\color[rgb]{0.133,0.545,0.133},
stringstyle=\color[rgb]{0.639,0.082,0.082},
morecomment=[l]{//},
morecomment=[s]{/*}{*/},
morecomment=[l]{\#},
morestring=[b]",
morestring=[b]',
}
\lstset{literate=
  {á}{{\'a}}1 {é}{{\'e}}1 {í}{{\'i}}1 {ó}{{\'o}}1 {ú}{{\'u}}1
  {Á}{{\'A}}1 {É}{{\'E}}1 {Í}{{\'I}}1 {Ó}{{\'O}}1 {Ú}{{\'U}}1
  {à}{{\`a}}1 {è}{{\`e}}1 {ì}{{\`i}}1 {ò}{{\`o}}1 {ù}{{\`u}}1
  {À}{{\`A}}1 {È}{{\'E}}1 {Ì}{{\`I}}1 {Ò}{{\`O}}1 {Ù}{{\`U}}1
  {ä}{{\"a}}1 {ë}{{\"e}}1 {ï}{{\"i}}1 {ö}{{\"o}}1 {ü}{{\"u}}1
  {Ä}{{\"A}}1 {Ë}{{\"E}}1 {Ï}{{\"I}}1 {Ö}{{\"O}}1 {Ü}{{\"U}}1
  {â}{{\^a}}1 {ê}{{\^e}}1 {î}{{\^i}}1 {ô}{{\^o}}1 {û}{{\^u}}1
  {Â}{{\^A}}1 {Ê}{{\^E}}1 {Î}{{\^I}}1 {Ô}{{\^O}}1 {Û}{{\^U}}1
  {œ}{{\oe}}1 {Œ}{{\OE}}1 {æ}{{\ae}}1 {Æ}{{\AE}}1 {ß}{{\ss}}1
  {ç}{{\c c}}1 {Ç}{{\c C}}1 {ø}{{\o}}1 {å}{{\r a}}1 {Å}{{\r A}}1
  {€}{{\EUR}}1 {£}{{\pounds}}1
}

\def\ContinueLineNumber{\lstset{firstnumber=last}}
\def\StartLineAt#1{\lstset{firstnumber=#1}}
\let\numberLineAt\StartLineAt

%%%% CUSTOM COMMANDS %%%%

\newcommand{\figref}[1]{Figure \ref{fig:#1}}% ~\figref{figurnavn} IKKE {fig:figurnavn}
\newcommand{\tabref}[1]{Table \ref{tab:#1}}
\newcommand{\chapref}[1]{Chapter \ref{cha:#1}}
\newcommand{\charef}[1]{Chapter \ref{cha:#1}}

% ¤¤ Visuelle referencer ¤¤ %
\usepackage[colorlinks]{hyperref}			 	% Giver mulighed for at ens referencer bliver til klikbare hyperlinks. .. [colorlinks]{..}
\hypersetup{pdfborder = 0}							% Fjerner ramme omkring links i fx indholsfotegnelsen og ved kildehenvisninger ¤¤
\hypersetup{														%	Opsætning af hyperlinks
    colorlinks = false,
    linkcolor = black,
    anchorcolor = black,
    citecolor = black
}
\setcounter{tocdepth}{1}
\makeatletter
\renewcommand{\l@section}{\@dottedtocline{1}{1.5em}{3.0em}}
\makeatother
