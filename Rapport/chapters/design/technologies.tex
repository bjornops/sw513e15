\section{Components (?)}
Chapter \ref{cha:technologies} examined the various components viable for use in the nodes. This section contains the decisions made regarding the components.

Both Arduino Uno and Mega are used in the solution, because having specifications good enough for the project and their availability to the project group. The Mega boards are not faster, but can contain more program code and allows for more components. This might not be necessary, but the devices are used as they were available.

The Raspberry Pi B+ model is used in the solution as the receiver and handler of data. This device has the capability to create user interfaces and show the data received from the sensors. The B+ model is used as it was available.

The data that is chosen to monitor with sensors in the solution is the moisture of the ground. This have been chosen as the two alternatives, pH and compaction, are changing at a slower pace, and is way more expensive to include in every sensorunit. To get data about the pH and compaction of the ground a manual check would make more sense, using another solution. \todo{check om det her er for farvet.}

The device chosen for this project is the RF24 because of the price. The shorter range is not a problem, as the project is more of a proof-of-concept. The speed is also good enough for the requirements, which makes this module good for the product described in the report.



\section{Network and protocol}
