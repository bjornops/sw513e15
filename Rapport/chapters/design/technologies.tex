\section{Components}
Chapter \ref{cha:technologies} examined the various components viable for use in the nodes. This section contains the decisions made regarding the components.

Both Arduino Uno and Mega are used in the solution, as they qualify regarding specifications, and due to their availability to the project group. The Mega boards are not faster, but can contain more program code and allow for more components. Although not necessary, but the devices are used as they are available.

The Raspberry Pi B+ model is used in the solution as the main node and handler of data\todo{why?}. This device has the capability to display and power user interfaces and show the data received from the sensors. The B+ model is used as it is available to the project group.

The data that is chosen to monitor with sensors in the solution is the moisture of the ground. This has been chosen, as the two alternatives, pH and compaction, are changing at a slower pace, and is more expensive in regards to sensors. To get data about the pH and compaction of the ground a manual check would make more sense, using another solution. \todo{check om det her er for farvet. Ja, det er det. Hvorfor bedre at måle manuelt? Begrund. Motivér.}

The device chosen for this project is the RF24, while it is sufficient, and because of its documentation, price and low power usage. The shorter range is not a problem, as the project is more of a proof-of-concept\todo{dette står da ingen steder!?}. The data rate\todo{synonym: bandwidth, something} is also good enough for the requirements, which makes this module good for the product described in the report.



\section{Network and protocol}
