\section{Components}
Chapter \ref{cha:technologies} examined the various components usable in the nodes. This section contains the decisions made regarding the components.

\subsection*{Arduino - sensor nodes}
%Explain the usage of Arduino Mega
Both Arduino Uno and Mega are used in the solution, as they qualify regarding specifications, and due to their availability to the project group. The Mega boards are not faster than the Uno boards, but can contain more program code, more data and have more pins available. While the solution should be runnable on the Arduino Uno boards, the solution is restricted to the Uno's capacity, extra memory and pins on the Mega boards are superfluous. The Mega boards are still used in the project, not because of the extra functionality, but to enable the project group to test a larger system of nodes.

\subsection*{Raspberry Pi - main node}
The Raspberry Pi B+ is used in the solution as the main node, based on the need to store, handle, and present data. This device has the responsibility of displaying the user interface, as well as processing and presenting the data received from the sensor nodes, as it has more processing power than the Arduinos.


\subsection*{nRF24L01 Transceiver}
The communication device chosen for this project is the nRF24L01, because of its documentation, and low power consumption. The transfer speed of the module is sufficient to send the small data packets, which makes this module suitable for the solution.

The nRF24L01 module contains multiple features for detecting packet loss. These includes checking hashes and sending acknowledgments\cite{nf24datasheet}. These features makes it harder to replace the radio modules in the nodes, as they are platform specific to the nRF24L01. The analysis examined the possibility of using different radio modules in case some requirements change, which is not possible if using platform specific features.

These features have been disabled, which means that the radio packets does not contain a checksum, nor does it send the default acknowledgments when packets arrives. The speed and output power have also been decreased, so the module will transfer slower, and have increased range.

\begin{figure}[!h]
	\centering
	\makebox[\textwidth][c]{\includegraphics[width=1\textwidth]{figures/Raspberry-Arduino-Tree.png}}
	\caption{Raspberrry Pi main node and Arduino Uno sensor nodes wirelessly connected in a tree.}
	\label{fig:raspbuinoTree}
\end{figure}

\subsection*{Setup}
With the components chosen and the data sheet for the nRF24L01 examined \cite{nf24datasheet}, the sensor nodes will be assembled as seen on figure \ref{fig:compsketch}. The full system will consist of a single Raspberry Pi as the main node and multiple Arduinos as sensor nodes. An example of the full system can be seen in \figref{raspbuinoTree}.

\begin{figure}[!h]
	\centering
	\makebox[\textwidth][c]{\includegraphics[width=1\textwidth]{chapters/design/figures/Sketch_bb.png}}
	\caption{Arduino connected to sensor and radio module.}
	\label{fig:compsketch}
\end{figure}

