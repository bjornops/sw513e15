\section{Data}
The final solution will monitor the status of the determined properties of the golf course. The data observed should be stored and treated, to be usable to the greenkeeper in maintaining the golf course. Data from previous requests should be accessible later, making it necessary to save this data somewhere. To access this data, some kind of user interface is also necessary. The data should be presented in a way that greenkeepers can use it, without requiring them to read a manual or otherwise study the system. 

The saved data could be useful at some later date, for example in keeping  statistics of the properties being watched.

Because of the relatively small platforms being used in the sensor nodes, it is not possible to save all data on each node. This makes it the main nodes' job to save and present the collected data.

The solution chosen for this, is that all data is saved on the main device, ordered by date and time, and then accessible through a user interface. The user interface is run as a website, and will be accessible from any platform by connecting to the main node.

This means that nodes in the system will only have the task of acquiring data and relaying it towards the main node. All storage and processing of data will be handled on the main node.

\subsection{Transfer speed}
Faster transfer speed uses more energy, but in a shorter amount of time. Slower transfer speed uses less energy, but over longer time for the nRF24L01. The slower data transfer rate has a significantly longer reach for the nRF24L01, and will therefore be implemented. 

\subsection{Pairing}
In order to add a new node to the network, the node is first needed to be paired up with the main node.
This way, a node could have assigned a unique identity to be able to recognize its sensor readings from other sensors. 
The protocol is described more indepth in \ref{cha:floodingSec}
