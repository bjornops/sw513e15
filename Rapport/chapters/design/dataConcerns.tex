\section{Considerations}
Considerations about collected data storage and power consumption for the final solution is elaborated in this section.

\subsection*{Data concerns - We do not know version}\todo{purpose of this section?}
The final solution will monitor the status of the determined properties of the golf course. The data observed should be stored and treated somehow.\todo{rephrase/elaborate}

The solution is supposed to assist the greenkeepers in their daily work. To actually reduce the greenkeepers workload, the final solution should have a main node that gather and presents the data, so that measures can be made to maintain the golf course.

%det ser skidt ud at dele linjer sådan, men det hjælper på commit fuck-ups :)

This\todo{what is "this"} also makes it unnecessary to keep long term data on the non-main nodes nodes.\todo{why?} 
Each node should only have the responsibility\todo{synonym} of acquiring data from its sensor, transmit the data to the main node. 
Storage and processing of the data should only be handled by the main node. 
This way, each node\todo{Suggestion: "device in system" or "node in network"} in the system will require less processing power and will also be replaceable\todo{why is it replaceable?}, except for the main node.

\subsection{Transfer speed}
Faster transfer speed uses more energy, but in a shorter amount of time. Slower transfer speed uses less energy, but over longer time for the nRF24L01. The slower data transfer rate has a significantly longer reach for the nRF24L01, and will therefore be implemented. 

\subsection{Pairing}
In order to add a new node to the network, the node is first needed to be paired up with the main node.
This way, a node could have assigned a unique identity to be able to recognize its sensor readings from other sensors. 
The protocol is described more indepth in \ref{cha:floodingSec}
