\section{Design intention}
This section contains the description of the system's intended design and usage. The design will be based on the requirement specification in \charef{requirements}. The core function of the solution is to transfer data wirelessly from several sensor nodes to a main node, and to be used by a person working on a golf course. 

The human aspect has requirements regarding aesthetics and user interface design, but that is secondary to the problem solving task of gathering data. This section will mainly consider the functional requirements to develop a functioning system, whereas data analysis and user experience with the solution are proposed as further work in another project. Nevertheless, a user must be able to utilize the system, and the main characteristics of the solution will be described in the next section.

\subsection{Workflow}\label{cha:workflowDesign}
% Practical use of the solution
As the solution's purpose is to support in monitoring a golf course, the solution should make the task less time consuming. The intention is that the data should be gathered and updated on-demand, so that the golf course status is sufficiently new for the greenkeepers' usage.

The solution is designed to function according to the following procedure:
\begin{enumerate}
	\item The system user will use an simple interface to demand readings from the sensors in the network.
	\item The main node responds to the demand by broadcasting a request signal in the network.
	\item The network nodes receives the request signal, gathers data from its sensors and sends it back to the main node before creating a new request signal to broadcast.
	\item The main node collects all data and append each reading to the respective origin node in a data sheet made available to the user.
	\item The user can evaluate further action based on the results.
\end{enumerate}

An on-demand solution has an advantageous property. The system components can save energy by not updating data unnecessarily, and the user is able to evaluate the age of the last measurements and decide whether to request new data or decide if the previous is sufficient. The on-demand solution, although, relies on the time required to gather the sensor readings to be useful. The current reference point is the time it takes to manually check the golf course status. Therefore should the data collecting time be strictly less than time used doing it manually.

A fully automated solution, means that the system could be set to gather data periodically, so that sensor readings are available once or multiple times during a given time frame. This ensures that the data is always up to date. An automated solution, however, has the disadvantages that the system could waste energy on readings that are not needed, or not having readings available when needed, if the time frame is incorrect.

The solution is deemed on-demand, because of the flexibility it provides. It can also be automated, by simulating a user request at a given time, in addition to let the users request readings whenever desired.


\subsection{Data handling and storage}
% Save the data!
The readings performed by the sensor nodes are the data used in the solution. The solution will monitor the status of the determined properties of the golf course. The observed data needs to be stored and processed to be usable for the greenkeeper in the process of maintaining the golf course. Data from previous requests should be accessible later, making it necessary to store previously received data. To access this data, a user interface is necessary. 

% How and where to save data: not nodes
There exists several methods to request and receive data. One method could be to request data from a single specific node in the network, evoking every node along the branch to the node. The other method is to request data from every node at once, evoking the entire network, which is the method chosen to implement. In addition, because of the limited amount of memory available in the sensor nodes, it is not possible to save data about every other node in the system as well as all the sensor readings on each node. This restricts the historical data to the amount of readings accommodable on the sensor nodes. A solution to the unnecessary communication as well as the restriction of stored data is to use the main node as a storage and user interface device\todo{unnecessary communication?}.

% How and where to save data: main node!
Storing the data on the main node, requires a sufficient amount of memory to be able to save a significant history of data. Handling and representing the data requires processing power. A computer, in the form of a Raspberry Pi, is capable of fulfilling the role as the main node. 

% Refresh the choice of moisture sensor / What data to monitor
The data to be monitored with sensors in the solution, is the moisture in the ground. This choice was made during chapter \ref{cha:sensorChapter}, but as before mentioned, any sensor which returns a digital value can be used.

% Conclusion
The sensor nodes task in the system is reading its sensors and transfer the data to the main node. All storage and processing of data will be handled on the main node.
The stored data can be used for statistics and other purposes to further optimize the golf course maintenance, but that is considered beyond this project's objective.

% Data representation
The user interface is run as a website, and will be accessible from any platform by connecting to the main node directly via its IP.