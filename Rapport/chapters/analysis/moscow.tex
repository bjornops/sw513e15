\chapter{MoSCoW}
This chapter aims to map the requirements of the project and sort them based on different priorities. First a listing with the requirements and their priorities, followed by a description for contents of each category.

Requirements are an essential part of developing a system that fits its purpose. All requirements can be considered important, but that they are all required for the systems purpose might not be case. By categorizing the requirements it becomes possible to identify and differentiate the important criteria from the lesser ones. 

The MoSCoW (Must have, Should have, Cant have and Wont have) analysis method have been used to sort the requirements. The requirements have are the result of the this reports analysis chapter.


Must have:
\begin{itemize}
\item[Transfer data from all sensor nodes to a destination/main node wirelessly]
\item[Implement a network with a communication protocol]
\item[The system should respond appropriately to a disconnecting node.]
\item[Nodes equipped with a moisture sensor.]
\end{itemize}

Should have:
\begin{itemize}
\item[Be able to seamlessly add and remove sensors to an existing system (Hot-pluggging)]
\item[Be modular build, so parts of a node can be replaced if destroyed]
\end{itemize}

Could have:
\begin{itemize}
\item[Separate system installations, so nodes do not connect to another installation nearby golf course]
\item[Have a graphical user interface for presentation of data.]
\end{itemize}

Wont have:
\begin{itemize}
\item[PH sensor and pressure sensor]
\end{itemize}

%In-depth description goes below:

Must have:
\item Transfer data from all sensor nodes to a destination/main node wirelessly
Being able to transfer data between the nodes is the core functionality for the system


%Implement a network with a communication protocol


%The system should respond appropriately to a disconnecting node.


%Nodes equipped with a moisture sensor.



Should have:
%Be able to seamlessly add and remove sensors to an existing system (Hot-pluggging)


%Be modular build, so parts of a node can be replaced if destroyed



Could have:
\item Separate system installations, so nodes do not connect to another installation nearby golf course
If the final system includes the hot-pluggable functionality, the individual nodes should stay connected to a particular installation and not be confused by another nearby installation of the system. This requirement could potentially become very complex, and the problem with two installations that interfere with each other might not be relevant if a single installation cover the entire golf course.

\item Have a graphical user interface for presentation of data.

Wont have:
\item PH sensor and pressure sensor








