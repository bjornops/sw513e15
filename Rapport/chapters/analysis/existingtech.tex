\section{Existing technologies}
This section will contain an analysis of existing technologies used for golf course maintenance. The aspects covered are according to the interview answers.

\subsection{RainBird}
Øeh.. Well fuck.

\subsection{Toro Turf Guard\texttrademark{}}
The company Toro created the Turf Guard\texttrademark{} sensors that allows greenkeepers to keep track of sensor data in real time. The sensors measure moisture, temperature, and salinity in the soil\cite{turfGuard}.
The sensors are wireless and has a range of 152 meters, but can be connected to repeaters to get a range of 1500 meters\cite{turfGuard}. It runs on the 900MHz ISM band, which requires no registration.

Turf Guard\texttrademark{} sensors are clever, as they contain two layers of sensors, allowing them to measure in the top soil and further down with one device.

\begin{wrapfigure}{r}{0.4\textwidth}
\begin{center}
\includegraphics[width=0.3\textwidth]{chapters/analysis/figs/Turfguard.png}
\caption{Toro Turf Guard\texttrademark{}.}
\label{fig:arduinouno}
\end{center}
\end{wrapfigure}



The size of device allows it to fit into a standard hole-size, making them easy to install on greens where a tool for creating holes are used\cite{turfGuard2}.

The entire system can be monitored from a computer, where the data are transmitted and shown in a user interface with locations for sensors available\cite{turfGuard2}.

The system requires the usage of other Toro systems to be working, making it hard to implement if systems from another manufacturer is already used. It also requires the use of repeaters nearby the sensors, for transmitting data to the main device\cite{turfGuard2}.
