\chapter{Use case}
The purpose of this project is to create a protocol that allows multiple Arduinos to share data to a single endpoint, but a use case is needed to test this protocol.

The chosen use case for this report is soil moisture for use on golf courses. A golf course is usually very large, and covering an entire golf course with cords would be a big task. Furthermore this would make the system hard to extend or hot pluggable.

That makes this project a good use case for golf courses, as soil moisture is important in determining where it is necessary to water the course. Wasting water is not much of a problem in Denmark, but some places in the world, water is a sparse resource. Using less water on a large golf course could save money down the line too, and is good for the environment. The use of radiocommunication is well suited for large, open spaces, like a golf course.

\section{Interview}
An informal interview was performed with a greenkeeper at Aalborg Golfcourse, Kim Jensen. Kim provided insights on the requirements of the system, potential problems, and ideas for later iterations of the system. The interview questions and answers can be found in Appendix \#\#. 

Some holes on Aalborg Golfklub is located in a swampy area, which often makes watering them obsolete. This project could help stop some of this water waste. Some holes are in a moor, where watering is often required. When it is required, this project could inform the greenkeepers, who could act on this.
Making the product hot pluggable would be useful if there is suddenly more or less water than usually, in which case the product can be added to this location, and the water levels can be measured to determine if there is a problem, and in that case how to solve it. The same applies if an area is more dry than usual.

An important consideration is where to place the devices. The sensors needs access to the soil in order to measure the moisture levels, but they should not be placed above ground as golfers could hit and destroy the devices. The initial idea was to place it in the holes for the sprinklers, but this is bad since sprinklers often leak, and the sensor would therefore get inaccurate readings.
This makes it necessary to bury the devices at known locations in the course. The depth the devices are buried at is important, due to the different types of grass. The greens use a 2cm. layer of sand at the top, which allows water to quickly go through. This makes it necessary to bury the devices 10-15cm. in the ground. These requirements changes based on the kind of grass and what is beneath.

Regarding future iterations of the project, Kim suggests adding a pH meter to the devices. This would allow the greenkeepers to create specific mixes of fertilizer for different parts of the course, depending on the pH value of the soil. 