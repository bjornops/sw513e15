\section{Interview}
Two interviews were performed with a greenkeeper at Aalborg Golfcourse, Kim Jensen. The first being an initiating interview, and the second being a more in-depth interview. The interviewee provided insights on ideas for the system, requirements, potential problems, and additions for later iterations of the system. Everything presented in this interview is either information from the interviewee or conclusions of the information. The exact interview questions and answers can be found in Appendix \ref{cha:interviewKim}. 

In the early interview, the interviewee provided the initial idea of automating the gathering of data about the golf course. He suggested soil moisture, pH value in the soil, and pressure. These properties are currently monitored manually, and could be automated to save time for the greenkeepers.

Making the system easily scalable would be useful if there is suddenly more or less water than usual, in which case devices could be added to a new area, and the water levels could be measured more specifically.

The interviewee mentions that a wireless system would be preferred over a wired system. At Aalborg Golfklub, the ground of the golf course already contains wires for both the lights and the watering system. A wireless solution would be simple to implement compared to a wired system. Furthermore, no risk of cutting an important wire for the system would exist. Also, a wireless system would be more simple to scale or restructure than a wired system.

Another consideration is where to place the devices. The sensors needs access to the soil in order to measure the moisture levels, but they should not be placed above ground as golfers could hit and damage the devices. This makes it necessary to bury the devices at known locations in the course. This is an important consideration because burying electronics in a humid environment alone is a challenge. Furthermore, if the system should be wireless as preferred, considerations about antenna quality should be made.

The depth the sensors are buried at is important, due to the different types of grass. The greens use a layer of sand at the top, which allows water to quickly go through. This makes it necessary to place the sensor at around 10-15cm. in the soil. To avoid wiring between the node and the sensor, the whole device should buried at that depth.

In the interview, the interviewee mentions several other possibilities for the project to save even more work. The pH-meter would allow the greenkeepers to create specific mixes of fertilizer for different parts of the course, depending on the pH value of the soil. The pressure sensor could help determining where on the course the soil has to be pricked, in order to keep the grass growing and water flowing.