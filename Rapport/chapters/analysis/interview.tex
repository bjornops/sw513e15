\section{Interview}\todo{Unclear what information is from the interview and what's deduced info}
Two interviews were performed with a greenkeeper at Aalborg Golfcourse, Kim Jensen. Kim provided insights on ideas for the system, requirements, potential problems, and additions for later iterations of the system. The interview questions and answers can be found in Appendix \ref{cha:interviewKim}. 

Kim provided the initial idea of automating the gathering of data about the golf course. He suggested soil moisture, pH value in the soil, and pressure. These properties are currently monitored manually, and could be automated to save time for the greenkeepers.

Some holes on Aalborg Golfklub are located in a swampy area, which often makes watering them obsolete\todo{Are these watered today? Specify.}. This projects solution could help moderate this water waste. Some holes are in a moor, where watering is often required. When it is required\todo{what? why? how come?}, a solution could inform the greenkeepers, who could react to this.

Making the devices\todo{which devices?} hot pluggable\todo{define.} would be useful if there is suddenly more or less water than usual, in which case a new device can be added to this location, and the water levels can be measured to determine if there is a problem, and in that case how to solve it \todo{rephrase this sentece.}. The same applies if an area is more dry than usual.

An important consideration is where to place the devices\todo{not our business}. The sensors needs\todo{source} access to the soil in order to measure the moisture levels, but they should not be placed above ground as golfers could hit and damage the devices. The initial idea\todo{spawned from where?} was to place it in the holes for the sprinklers, but this could be unfortunate if the sprinklers leak, and cause the sensor to acquire inaccurate readings. This makes it necessary to bury the devices at known locations in the course\todo{does the device need to be buryed or just sensors?}. The depth the devices are buried at is important, due to the different types of grass. The greens use a layer of sand at the top, which allows water to quickly go through\todo{move to golf course analysis?}. This makes it necessary to place the sensor at around 10-15cm. in the soil.

In future iterations\todo{How you know?} of the project, the other sensors mentioned in the interview could be implemented to save even more work. The pH-meter would allow the greenkeepers to create specific mixes of fertilizer for different parts of the course, depending on the pH value of the soil. The pressure sensor could help determining where on the course the soil has to be pricked, in order to keep the grass growing and water flowing.