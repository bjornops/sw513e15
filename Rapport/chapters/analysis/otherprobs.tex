\section{Other issues}
This section contains information about other potential issues that have not been directly encountered through the analysis, but still could be relevant for the final system.

\subsection{Power consumption and range consideration} \label{cha:batcons}
To be able to transmit data wirelessly, from all areas of a golf course to a node, the wireless range of the devices is considered.

Modules with a long range can require more power than a short range module, which would require more of a power supply. Furthermore, the modules could physically become larger, because of a bigger antenna, which would increase the amount of work in burying the devices in the ground. Because of the risk of being hit, and destroyed, by either a golf ball, or a lawn mower, it is not an option to place the entire device above ground. Using big antennas would make it easier to establish a direct wireless connection between nodes.

Another approach with short range modules can be considered. Short range radio modules usually requires less power than a long range module, thereby lowering the requirements of the power supply. It would also require less work to bury a device with a small antenna, compared to a device with a big antenna. A problem in such a solution is the radio module range. If some devices were to be placed on a far corner of a golf course, the devices might not be able to transmit its data to a central node. 

\subsection{Security}
This section contains the security considerations regarding unauthorized access, data loss or theft.

The main function of the system is to transmit data wirelessly through a network. Since the data is sent wirelessly, it can be intercepter or interfered resulting in incorrect data.

A problem could be the process of connecting a new device to the network. If a device is connected without any authentication, it could become a problem if a neighbouring network is using the same frequency which could result in the two networks exchanging data unintentionally. A malicious device could also inject data into the system.

Considering the hardware design of a device, it would also be relevant to analyze if the device can be physically reconstructed to intentionally transmit malicious data to the network. Though being more a question of physical security, by preventing access to the device, this could still be relevant.

The major issue in the system is that data is transmitted wirelessly and it is therefore exposed to malicious input. This should be taken into account in the design of a final solution.

\subsection{User interface}
As the main purpose of the system is to transfer data from sensors to an end node, where a user can read and react appropriately, it would be necessary to present this data for the user. Since it is practically impossible for humans to decipher bytestreams, a user interface with a more simple way of presenting the data would be practical, by giving a better overview of the data.

An excessive amount of interactions with the user interface would be unnecessary, as the system would only serve as a way for reading and understanding the collected data.
