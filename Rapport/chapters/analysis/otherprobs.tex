\section{Other issues}
Some other potential issues not directly encountered through the analysis, but which could still be relevant, will be presented here.

\subsection{Security}
In most softwaresystems it is important to evaluate the security, to avoid unauthorized access or data loss/theft. This project is not excluded from such evaluation, and in this section the greatest security issues will be considered.

The main functions of this project is to exchange data throughout a network wirelessly. Since the data is sent wirelessly, it can easily be manipulated or interfered resulting in incorrect data.

Another view is the process of connecting a new device to the network. If a device is connected too easy, it could become a problem if a neighbouring network is using the same protocol which could result in the two networks exchanging data unintentionally. Another point is that a malicious device could easily inject data into the system.

Considering the hardware design of a measuring device, it would also be relevant to analyze if a device can be physically reconstructed to intentionally transmit malicious data to the network. Though being more a question of physical security it could be an aspect in the design phase.

There could potentially be a lot of minor security issues, but the major issue in this project is that data is exchanged wirelessly and it is therefore very adaptive to malicious input. This must be taken into account in the design of a final solution.

\subsection{User interface}
As the problem is to transfer data from sensors to some point, where a user can read and react appropiatly to it, it would be necessary to present this data for the user. As the typical human being is unable to read bytestreams, a userinterface with some more informal way of presenting this data is a practical thing.

The amount of interaction would not need to be excessive, as the system would only serve to feed the user with data, and as such a simple interface for oneway communicating the data to the user could be enough.
