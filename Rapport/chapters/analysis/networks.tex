This section will contain descriptions of networks and network theory, and connect to the established use case.

A computer network is a collection of computers and devices connected so that they can share information and services \cite{mansfield2009computer}. The way these devices are interconnected is called topology. The communication structure the devices use to exchange information over a medium is called protocol, which will be described in another section.

In this section the term node is used for a device connected to the network, to avoid binding to a specific device type.

There are different types of network topologies, and here are some examples:
\begin{itemize}
	\item Ring
	\item Line
	\item Bus
	\item Tree
	\item Star
	\item Mesh
	\item Fully connected
\end{itemize}

These can be seen in figure xx, that will be put here somewhere. %The topologies considered for this use case are star, mesh, tree and ring networks. 

The star network has one main node that the other nodes are directly connected to. An example of a star topology network is wifi, typically with a wireless router to which other devices connect to gain network access. The wireless router will handle all the network communication and redirect the information to the correct device. A limitation of the star network is that all network devices must have a connection to the main node, and therefore be clustered within the reach of the main node coverage. \cite{todo}

A tree network also utilizes a main node, but the devices in the network do not necessarily connect directly to the main node, but rather connect to another node that relays to the main node. This can repeat over multiple levels, so that information is relayed through several nodes, before reaching the main node. The tree network has a fixed node structure, and the relay nodes will route the information towards the destination. \cite{todo}

Another topology is the mesh network. It is a type of mobile ad hoc network. There are two kinds of mesh networks, the full-mesh and the partial-mesh networks. A full-mesh describes a network where all the nodes are interconnected, similar to a fully connected graph. A partial mesh is also a mesh network, but does not require all nodes to be connected, so that it's similar to a tree network with cycles. The mesh networks have the same limitation as a tree network, regarding the information transmission delay because the information transmits through up to several nodes. \cite{g2wmn}

The best fitting network topology for the use case is a mesh network. It can transmit information through the network without limiting the connected devices to a certain distance from a main device, as with a star topology. It is also capable of multiple methods of distributing information. It does not rely on all nodes working at all times, as the network can reconfigure and find another path of information. This applies as long as there somehow exists another node that can relay the information towards the main node.

Move this paragraph?
There are multiple methods of communicating through a mesh network, therein routing and flooding are two alternatives. Routing will transfer the information towards a destination node, whereas the flooding method will notify all nodes within reach to distribute the information forward, and this will repeat until all nodes has transmitted the information, and hence the destination node also has received the information.

%''\textit{An ad-hoc wireless network is a wireless network, comprised of mobile computing devices that use wireless transmission for communication, having no fixed infrastructure}'' - \cite{murthy2004ad}. 