This section will contain descriptions of networks and network theory, and connect to the established use case.

A computer network is a collection of computers and devices connected so that they can share information and services \cite{mansfield2009computer}. The way these devices are interconnected is called topology. The communication structure the devices use to exchange information over a medium is called protocol, which will be described in another section.

There are different types of network topologies, and here are some examples:
\begin{itemize}
	\item Ring
	\item Line
	\item Bus
	\item Tree
	\item Star
	\item Mesh
	\item Fully connected
\end{itemize}

These can be seen in figure xx, that will be put here somewhere. //The topologies considered for this use case are star, mesh, tree and ring networks. 

The star network has one main node that the other nodes are directly connected to. An example of a star topology network is wifi, typically with a wireless router to which other devices connect to gain network access. 

A tree network also utilizes a main node, but the devices in the network do not necessarily connect directly to the main node, but rather connect to another node that relays to the main node. This can repeat over multiple levels, so that information is relayed through multiple nodes, before reaching the main node. 

Ring networks consists of a line of devices, where the end node reconnects to the first node. 



An ad-hoc wireless network is a wireless network, comprised of mobile computing devices that use wireless transmission for communication, having no fixed infrastructure \cite{murthy2004ad}. 