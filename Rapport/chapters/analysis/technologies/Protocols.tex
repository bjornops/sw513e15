A mesh network can use a wide variety of protocols, to manage the route data is transferred.
In networking, a protocol is a special set of rules and standards for how nodes would interacts with each other.
A well known protocols could be TCP/IP(Transmission Control Protocol/Internet Protocol), which today are used to communicate between almost anything with a internet connection.
The mesh network we are looking at is a radio based network, and therefore some more relevant protocols will be examined. 
A few excising protocols will be presented in this section.

\subsection{Time division multiple access}
Time division multiple access(TDMA) is protocol that divides a single channel into smaller time slots.
Each time slot transmits one byte or a segment of a signal, in a sequential serial data format.
Each slot is active of a small amount of time, before the next slot in the queue get time to transmit.

TDMA is as an example used in the T1 telecommunication transmission system.
Each T1 channels carry up to 24 voice telephone connections.
Where each connection covers 300 Hz to 3000Hz and is digitized at an 8-kHz rate, which is two times the highest frequency component needed to retain all the analog content.
\begin{figure}[!h]
	\centering
	\makebox[\textwidth][c]{\includegraphics[width=1\textwidth]{figures/TDMA.png}}
	\caption{Illustration of the TDMA protocol}
	\label{fig:TDMAfigure}
\end{figure}

On \figref{TDMAfigure} it is seen how the channel, for T1, is split up into 24 smaller pieces.
Each time slot is accountant for a user using a voice channel, to talk to some other user.
Each time slot is of the size 8-bit, the user is unaware of this small data size, and that other are using the same channel, because the shift between each time slot happens so fast.
This gives the illusion of each user talking with no interruptions, even though each user only is assigned 1/24 of the total bandwidth.
A single bit is used to synchronization.
The TDMA system can maximum achieved a data rate of 1.544 Mbit/second.\cite{TDMA}

TDMA can be used for any system that require several device to use the same channel, without interfering with each other.

\subsection{Ad hoc On-Demand Distance Vector Routing}
Ad hoc On Demand Distance Vector(AODV) routing algorithm is a routing protocol designed for ad hoc networks. 
It is an on demand algorithm, meaning that it builds routes between nodes only as desired by source nodes.\cite{AOVD1}

AODV is a network where only the local nodes around a newly added node, is affected by the addition.
If a link between nodes is broken, and it does not affect an ongoing transmission, no global notification occurs.
This leads to a network where new nodes easily can be added and removed without overhead on the entire network.
The transmission  route in AODV is managed so, that only nodes in the direct route is active, which reduce the need for route maintenance and reduce idle nodes.
The protocol can determine multiple routes between a source and a destination, but only a single one is implemented.
The route is not necessarily the shortest.

A downside to AODV is that if a single route breaks, due to a defect node, it is not possible to know whether other routes exists.
But if a route breaks, the protocol discovers a new route, if possible, as and when necessary.

When a node is ordered to send a packet to a specified destination, it checks its routing table to determine if it has a current route to the destination.
If there already exist a node, the packet will be delivered to the next node in the route, repeating until it arrives at the correct location.
If there does not exist a route, the node will initiates a route discovery process.

A route discovery process begins with the source creating a Route Request (RREQ) packet.
The packet contains the sources node's unique ID and IP, and the destination node's unique ID and IP.
The packet is then transmitted out from the source to it's neighbours.
The unique ID of each node is then stored in the packet, to ensure the same node is not transmitting more than once.
The destination ID is to ensure it knows when the destination node is reached.
A simple AODV network can be seen on \figref{AODVfigure}, where S and D represent the source and destination. It is visualized how the route discovery is invoked using the RREQ packet.\cite{AOVD2}

\begin{figure}[!h]
	\centering
	\makebox[\textwidth][c]{\includegraphics[width=0.4\textwidth]{figures/AODV.png}}
	\caption{Illustration of the AODV route discovery}
	\label{fig:AODVfigure}
\end{figure}


\subsection{Radio Link Protocol}
Radio Link protocol(RLP) is a automatic repeat request(ARQ)\footnote{An error-control method for data transmission that uses acknowledgements and timeouts} fragmentation protocol used over a wireless air interface.
Most air interface protocols have a packet loss of up to 1\% which is intolerable when handling sensitive data.
RLP detects losses in packets and with a retransmission tries to bring down the losses.
The retransmission can bring the loss down to 0.1\% to 0.0001\%.
This loss rate is more tolerable when handling sensitive and precise data.

RPL cannot request a certain payload size from the air interface, the air interface scheduler instead determines the packet size, based on changing channel conditions constantly.
Most of the other fragmentation protocols, such as 802.11b\footnote{An wireless networking specification} and IP, determine a payload of a certain size by the upper layers, and call upon the MAC.
These protocols are not as flexible as RLP, and sometime fail transition during small fades in a wireless environment.\cite{MobileComm}

RLP is used to make a more fail-safe environment for the transmitted data, and to ensure nothing is lost on the way.
The Radio Link Protocol is typically used in cellular transmission.