\section{Power supply}

%			NEW
Each node in the network will require power in order to function. There are multiple ways of powering the nodes, each with their own strenghts and weaknesses. Some of the relevant possibilities are the use of power cables, solar panels, and batteries. Each of these possibilities are discussed throughout this section.


%	POWER cables	%
\subsection*{Power cables}
Power cables could be used as a steady power source for network nodes. The cables would be dug down throughout the golf course whereof each node would get connected. This solution is ideal since every node in the network will always have their required power at hand, so running out of power will never be an issue. The power cables are therefore a solid long term way of providing power.

The requirement of the final overall solution that states that the nodes must communicate wirelessly becomes redundant if power cables were to be utilized. If the golf course would need to be dug up in order to lay down the power cables, cables for communication could just as well be laid down to ensure reliable communication.

The permanent aspect of the power cables may not be bonus in the long run. The areas of the golf course that are of interest to the greenkeepers might change during the season. The power cable solution would not allow the re-positioning of the nodes since they are dependent of the cables. This also puts constraints on the scalability of the final solution, since the addition of extra nodes to the network requires that the golf course gets dug up again.

\subsection*{Solar panel}
Solar panels could be utilized to power the nodes of the network. Each node would be equipped with its own small solar panel and be self sufficient. This solution, just as the power cables, will not run out of power. The solar panels also gives the nodes portability/flexibility, so they can be moved around the golf course if need be, as well as adding scalability for not requiring to the golf course to be dug up.

Since solar panels need the sun for generating power, it panels need to be above ground and cannot be dug down together with the sensor. This is a negative since this puts the solar panels at risk of straying golf ball, golf players, and lawnmowers. This means that the solar panels puts restrictions on the node placement, because it needs a placement where it can generate power from the sun, but at the same be sheltered from golf balls and the like.

%	BATTERIES	&
\subsection*{Batteries}
Equipping each node in the network with its own battery is another possibility. Utilization of batteries adds portability/flexibility to the nodes in the network, as opposed to the power cable solution. Nodes with a battery can be moved around the golf course if needed, instead of being dependent of where the power cables have been dug down. This also adds scalability to the final solution since the extra nodes can be added to the network without needing to dig up the golf course.

Something to take note of is the lifespan of the battery. The power cables will keep delivering the required power as long as need be. The same goes for the solar panels. Batteries on the other hand cannot keep up with the two alternatives in this aspect. When he battery runs out, it will be required of a greenkeeper to manually replace it. This means that the usage of batteries adds a constraint to the entire system about power consumption, which would have to be taken into consideration when designing the final solution.

%	SOLAR	%

