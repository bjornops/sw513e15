\section{Communication devices}
In the following section, different types of communication devices relevant to the project are examined. The goal is to determine which communication type and device is best suited\todo{no, to name/explain stuff, decision made in Design} for the solution.

\subsection{Bluetooth}
Bluetooth is a wireless technology standard designed to transfer data over short distances\todo{source}. Bluetooth is often used in phones and computers, and can be used to connect\todo{that's it? [she said]} devices, such as the ones to be designed as the solution. \todo{fix den her linje, er clueless.}

The number of devices that can be on a Bluetooth network is almost unlimited, and the the built-in interference reduction makes the technology usable for the product\todo{what prodct?} \cite{bluetoothbasics}.
Bluetooth devices have to be paired in order to exchange data, which makes it hard to create a hot pluggable network of devices\todo{why? motivate}. This, and the short maximum range of 100m\cite{bluetoothbasics}, makes it less\todo{than what?} adequate for this project, as golf courses are large and these devices will have to cover great distances.\todo{remember: no choices in this chapter}

\subsection{XBee}
XBee is a radio module by Digi. Multiple versions of the XBee modules exists, with differences in power usage, range, built-in/external antenna, and some with processors \cite{sparkfunXbeeGuide}.
XBees often use the ZigBee protocol. The ZigBee protocol is made for low-power devices, such as the product described in this report\cite{zigbee}.

\todo{JESUS, CHRIST! NO FREAKING "EASY" IN THE REPORT}
XBees are fast and predictable, making them a good choice for a project such as this. They can cooperate with the Arduino platform using the serial pins. There also exists special XBee shields made for Arduino.

XBees are unfortunately\todo{uheldigvis?} also quite expensive, which makes them unfit for this project. In later iterations\todo{a solution has no "later iterations". Further developement, fx?} of the solution \todo{vi bør diskuterre om det er produkt eller løsning?}\todo{bjørn synes det er en løsning på nogens problem, ikke et generelt produkt}  they could be used if longer ranges became a requirement or it had to interface with other devices using the ZigBee protocol or XBee modules.\todo{why is longer range requirement a counter argument for expensive?}

\subsection{RF24 Transceiver}\todo{aren't there several RF24s? we got the nRF24L01-something?}
The RF24 transceiver module is a radio module used for exchanging data between modules of the same type\todo{is this necessarily true? Can also communicate with other units using the same frequency?}. As with most of the XBee modules, the RF24 uses the 2.4GHz band, which is license-free in the whole world\todo{source}. It is well documented, and multiple libraries exists for using the device with Arduino\todo{well-documented: still no source?}.

The RF24 is cheaper than the XBees, and has shorter range and is potentially slower, depending on the communication protocol.

The full RF24 specifications can be seen in appendix \todo{add this.}

% The RF24 allows packets of up to 32 bytes, and is fast enough 
