\section{Communication devices}
In the following section, different types of communication devices usable in the project is examined. The goal is to determine which communication type and device is best suited for the solution.

\subsection{Bluetooth}
Bluetooth is a wireless technology standard designed to transfer data over short distances. Bluetooth is often used in phones and computers, and can be used for connecting devices as the ones to be designed as the solution. \todo{fix den her linje, er clueless.}

The number of devices that can be on a Bluetooth network is almost unlimited, and the the built-in interference reduction makes the technology usable for the product\cite{bluetoothbasics}.
Bluetooth devices have to be paired in order to exchange data, which makes it hard to create a hot pluggable network of devices. This, and the short range of (at most) 100m\cite{bluetoothbasics}. makes it less viable for this project, as golf courses are large and these devices will have to cover great distances.

\subsection{XBee}
XBee is a radio module by Digi. Multiple versions of the XBee modules exists, with differences in power usage, range, built-in/external antenna, and some with processors\cite{sparkfunXbeeGuide}.
XBees often use the ZigBee protocol. The ZigBee protocol is made for low-power devices, such as the product described in this report\cite{zigbee}.

XBees are fast and predictable, making them a good choice for a project such as this. They are easy to work with on the Arduino platform, using the Serial pins. There even exists XBee shields made for Arduino.

XBees unfortunately are also quite expensive, which makes them unfit for this project. In later iterations of the product/unsure \todo{vi bør diskuterre om det er produkt eller løsning?}  they could be used if longer ranges became a requirement or it had to interface with other devices using the ZigBee protocol or XBee modules.

\subsection{RF24 Transceiver}
The RF24 transceiver module is a radio module used for exchanging data between modules of the same type. As with most of the XBee modules, the RF24 uses the 2.4GHz band, which is license-free in the whole world. It is well documented, and multiple libraries exists for using the device with Arduino.

The RF24 is cheaper than the XBees, although it has shorter range and, depending on the chosen protocol, is potentially slower.

The device chosen for this project is the RF24 because of the price. The shorter range is not a problem, as the project is more of a proof-of-concept. The speed is also good enough for the requirements, which makes this module good for the product described in the report\todo{FLYT TIL DESIGN}.


The full RF24 specifications can be seen in appendix \todo{add this.}

\subsubsection{Testing}\todo{should this be here?}
When transmitting data from one unit to another, knowing that the data is correct can be important. To get an idea of how much error is introduced in the communication, a bit error test can be performed, known as BERT from this point.

\textbf{About BERT}

The basic concept of BERT is to send a known datastream from one unit to another, and check how much it differs from the expected datastream. As the RF24 is a radio transceiver there's possible factors as range and noise. \todo{source}

The duration of a BERT should vary depending on the results. To get an exact result it would have to take an infinite amount of time, which of course is not feasible. But du to the nature of randomness, if only a small amount of errors are met, then it could just be a "lucky"\todo{formulate}  test, so the testing length would have to be sufficiently long for getting a general idea.

\textbf{Executing BERT} 

Noise generated by another transceiver near the recieving point.

The code running on the arduinos for the test is:

Sender:
Code sending some deterministic value

Reciever:
Code accepting some value, and checks verifyes it against same method it was generated by.
Keeps track of amount of correct and wrong values

\textbf{Result of BERT}

\missingfigure{some table with resulting data at different ranges and with/without added noise}

The table show that it may or may not be acceptable for a solution for the problem.

%primarysource http://www.radio-electronics.com/info/rf-technology-design/ber/bit-error-rate-testing-bert.php