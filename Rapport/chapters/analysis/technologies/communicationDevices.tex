\section{Communication devices}
In the following section, different types of communication devices relevant to the project are examined.

\subsection{Bluetooth}
Bluetooth is a wireless technology standard designed to transfer data over short distances\cite{bluetoothbasics}. Bluetooth is often used in phones and computers, but are suitable for embedded systems requiring wireless transfer of data.

The number of devices that can be on a Bluetooth network is almost unlimited, and the the built-in interference reduction makes the technology usable for the solution\cite{bluetoothbasics}.
Bluetooth devices have to be paired in order to exchange data, which makes it hard to create a hot pluggable network of devices, as all devices communicating will have to be paired together, and not just added to the network. 
This, and the short maximum range of 100m\cite{bluetoothbasics}, could provide problems with Bluetooth.

\subsection{XBee}
XBee is a radio module by Digi. Multiple versions of the XBee modules exists, with differences in power usage, range, built-in/external antenna, and some with processors \cite{sparkfunXbeeGuide}.
XBees often use the ZigBee protocol. The ZigBee protocol is made for low-power devices, such as the product described in this report\cite{zigbee}.

XBees are fast and predictable, making them a good choice for a project such as this. They can cooperate with the Arduino platform using the serial pins. There also exists special XBee shields made for Arduino.

XBees are quite expensive, which makes them unfit for this project. If the solution had to interface with other devices using the ZigBee protocol or XBee modules they could be implemented.

\subsection{RF24 Transceiver}\todo{aren't there several RF24s? we got the nRF24L01-something?}
The RF24 transceiver module is a radio module used for exchanging data between modules of the same type\todo{is this necessarily true? Can also communicate with other units using the same frequency?}. As with most of the XBee modules, the RF24 uses the 2.4GHz band, which is license-free in the whole world\todo{source}. It is well documented, and multiple libraries exists for using the device with Arduino\todo{well-documented: still no source?}.

The RF24 is cheaper than the XBees, and has shorter range and is potentially slower, depending on the communication protocol.

The full RF24 specifications can be seen in appendix \todo{add this.}

% The RF24 allows packets of up to 32 bytes, and is fast enough 
