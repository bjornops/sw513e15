\section{Communication protocols}
\todo{missing "abstract" metatext}
\todo{define source node, used much, but not clear what it is}
\todo{define packet}
A few existing, relevant protocols will be presented in this section.
\todo{we haven't settled on mesh so far}
A mesh network can use a wide variety of protocols, to manage the route data is transferred.
In networking, a protocol is a special set of rules and standards for how nodes would interacts with each other\todo{source}.
Well known protocols could be TCP/IP(Transmission Control Protocol/Internet Protocol), which today is used to communicate between almost anything with a internet connection\todo{fix english and correctness}.
The mesh network we are looking at is a radio based network, and therefore some more relevant\todo{more relevant than what?} protocols will be examined. 

\subsection{Time division multiple access}
Time division multiple access(TDMA) is an access method that divides a single channel into smaller time slots.
Each time slot transmits one byte or a segment of a signal, in a sequential serial data format.
Each slot is active a short time period, before the next slot in the queue get time to transmit\cite{TDMA}.

TDMA is, for example, used in the T1 telecommunication transmission system.\todo{relevance? what is T1?}
Each T1 channels carry up to 24 voice telephone connections, where each connection covers 300 Hz to 3000Hz and is digitized at an 8-kHz rate, which is two times the highest frequency component needed to retain all the analog content.\todo{not protocol approach. very specific implementation example.}
\begin{figure}[!h]
	\centering
	\makebox[\textwidth][c]{\includegraphics[width=1\textwidth]{figures/TDMA.png}}
	\caption{Illustration of the TDMA access method}
	\label{fig:TDMAfigure}
\end{figure}

On \figref{TDMAfigure} it is seen how the channel, T1, is split up into 24 smaller pieces.
Each time slot is accountant for a user using a voice channel, to talk to some other user.
Each time slot is of the size 8-bit, the user is unaware of this small data size, and that other are using the same channel, because the shift between each time slot happens so fast.\todo{relevance: 8 bit and user unawareness?}
This gives the illusion of each user talking with no interruptions, even though each user only is assigned 1/24 of the total bandwidth.\todo{aren't they talking without interruption unless the connection breaks? }
A single bit is used to synchronization.
The TDMA system can maximum achieve a data rate of 1.544 Mbit/second \cite{TDMA}.\todo{good or bad? Motivate:)}

TDMA can be used for any system that require several device to use the same channel, without interfering with each other.


%\subsection{Radio Link Protocol}
%Radio Link protocol(RLP) is a automatic repeat request(ARQ)\footnote{An error-control method for data transmission that uses acknowledgements and timeouts} fragmentation protocol used over a wireless air interface.
%Most air interface protocols have a packet loss of up to 1\% which is intolerable when handling sensitive data.
%RLP detects losses in packets and with a retransmission tries to bring down the losses.
%The retransmission can bring the loss down to 0.1\% to 0.0001\%.
%This loss rate is more tolerable when handling sensitive and precise data.

%RPL cannot request a certain payload size from the air interface, the air interface scheduler instead determines the packet size, based on changing channel conditions constantly.
%Most of the other fragmentation protocols, such as 802.11b\footnote{An wireless networking specification} and IP, determine a payload of a certain size by the upper layers, and call upon the MAC.
%These protocols are not as flexible as RLP, and sometime fail transition during small fades in a wireless environment.\cite{MobileComm}

%RLP is used to make a more fail-safe environment for the transmitted data, and to ensure nothing is lost on the way.
%The Radio Link Protocol is typically used in cellular transmission.

\subsection{Dynamic Source Routing protocol}
The Dynamic Source Routing protocol (DSR) is a simple routing protocol designed specifically for use in multi-hop\todo{is multi-hop defined?} wireless ad-hoc networks.\todo{source}
With DSR the network is completely self-organizing and self-configuring, requiring no administration or existing network structure.\todo{source and motivation}
The nodes in the network work together to forward packets to destinations that can be outside of the senders transmission range.\todo{definition of mesh/tree network? already done}
As nodes can be added and removed ad-hoc\todo{added and removed ad-hoc, means what?}, the protocol automatically determine and maintain the routing of the packets through the network.
Since the number or sequence of nodes needed to reach any destination may change at any time, the resulting network topology may be quite rapidly changing.\todo{what? no comprehende}
DSR is created to create a very low overhead\todo{define overhead} yet being able to react quickly to change in the network environment.
The protocol provides a highly reactive\todo{highly reactive sounds like rubidium} service to ensure successful delivery of data packets where node may be moving around, or other changes in the network occur\cite{DSR}. 

The DSR protocol is composed of two main mechanisms that allow the discovery and maintenance of the routes in the network.

- Route Discovery\todo{any reason for Capital Letter} is the mechanism by which a node S wishing to send a packet to a destination node D obtains a source route\todo{source route or just route, or what?} to D.
Route Discovery is used only when S attempts to send a packet to D and does not already know a route to D.

- Route Maintenance is the mechanism by which node S is able to detect, while using a source route to D, if the network topology has changed such that it can no longer use its route to D because a link along the route no longer works.\todo{rephrase}
When Route Maintenance indicates a source route is broken\todo{define broken. route not exist?}, S can attempt to use any other route it happens to know to D, or it can invoke Route Discovery again to find a new route for subsequent packets to D.
Route Maintenance for this route is used only when S is actually sending packets to D.

DSR is a source routing protocol where packets carries a header, an ordered list of nodes which is the route.
This explicit use of routing allows the sender to select and control the route for the packets it sends.
Source routing allow for load balancing, since the sender can create different routes out though the network, avoiding "heavy traffic".\todo{high throughput for one special node?}
It is also a guarantee that the routes used are loop-free, since a generated route never use the same node twice.
By including this source route in the header of each packet, other nodes forwarding or overhearing any of these packets can easily\todo{fix} cache this information for future use\cite{DSR}.
\todo{requirement that all nodes know the whole network topology? Puts high strain on node compiling the route for each packet?}

\subsection{Ad-hoc On-Demand Distance Vector Routing}
Ad-hoc On-Demand Distance Vector(AODV) routing algorithm is a routing protocol designed for ad-hoc networks. 
It is an on-demand algorithm, meaning that it builds routes between nodes only as necessary by source nodes \cite{AOVD1}.

AODV is a network where only the nodes within reach of a newly added node is affected by the addition.
If a link between nodes is broken, and it does not affect an ongoing transmission, no global notification occurs.\todo{comes suddenly. motivate or move. What global notification}
This leads to a network where new nodes easily\todo{fix} can be added and removed without overhead on the entire network.
The transmission route in AODV is managed so that only nodes in the direct route are active, which reduce the need for route maintenance and idle nodes\todo{why and how does it reduce maintenance and idle nodes?}.
The protocol can determine multiple routes between a source and a destination, but only a single one is implemented.
The route is not necessarily the shortest \cite{AOVD1}.\todo{which route?}

A disadvantage to AODV is that if a single route breaks, for example due to a defect node, it is not possible to know whether other routes exists.\todo{how come we don't know?}
But if a route breaks, the protocol discovers a new route, if possible, as and when necessary.\todo{rephrase. What is actually described here?}

When a node is ordered\todo{who's ordering? is that a part of the protocol?} to send a packet to a specified destination, it checks its routing table\todo{define routing table. Not mentioned before?} to determine if there currently exists a route to the destination.
If there already exist a route, the packet will be delivered to the next node in the route, repeating until it arrives at the correct location.
If there does not exist a route, the node will initiate a route discovery process \cite{AOVD1}.

A\todo{"A" or "The": all or just this protocol?} route discovery process begins with the source creating a Route Request (RREQ) packet.
The packet contains the sources node's unique ID and IP\todo{what's IP?}, and the destination node's unique ID and IP.\todo{how does it know the destination node?}
The packet is then transmitted from the source to it's neighbours.
The unique ID of each node is then appended to the packet, to ensure the same node is not transmitting more than once.
The destination ID is to ensure it knows\todo{it knows? What knows what?} when the destination node is reached.
A simple AODV network can be seen on \figref{AODVfigure}, where S and D represent the source and destination. It is visualized how the route discovery is invoked using the RREQ packet.\cite{AOVD2}

\begin{figure}[!h]
	\centering
	\makebox[\textwidth][c]{\includegraphics[width=0.4\textwidth]{figures/AODV.png}}
	\caption{Illustration of the AODV route discovery}
	\label{fig:AODVfigure}
\end{figure}

\subsection{Better Approach To Mobile Ad-hoc Networking}
Better approach to mobile ad-hoc networking (B.A.T.M.A.N) is a routing protocol for multi-hop, ad-hoc mesh networks based on AODV. 
B.A.T.M.A.N is said to solve some of the more typical problems with the classical routing protocols, which is not well suited for wireless ad-hoc networks.
Some of the problems are, such networks are unstructured, are based on an inherently unreliable medium and dynamically change their topology.\todo{rephrase. meaning lost :(}

When using the B.A.T.M.A.N algorithm, the approach is to divide\todo{share?} the knowledge about the most reasonable path between nodes in a network to all of the participating nodes.
Each node in the network only account for the best path to all other nodes in the system.
This results in the need for a global knowledge about local topology changes becomes unnecessary.
In addition, B.A.T.M.A.N has an event-based, but timeless\todo{what does timeless mean?}, flooding mechanism that prevents the occurrence of loops and limits the amount of topology messages flooding\todo{flooding prevents flooding? Unclear to me} the network.

Each node in the network transmits a broadcast message(OGMs)\todo{what's OGM an abbreviation for?} to inform every node within range about its existence.
The nodes within range, then re-broadcast the OGMs, to nodes within their range, about the existence of the original initiator of this message and so on.
Thus the network is flooded with originator\todo{que? Isn't originator and inventor/discoverer?} messages. OGMs are small, the typical raw packet size is 52 byte including IP and UDP overhead\todo{source}.
OGMs contain at least the address of the originator, the address of the node transmitting the packet, and a sequence number.

OGMs that follow a path where the quality of wireless connection is poor will suffer from packet loss or delay on their way through the network.
Therefore OGMs that travel on good routes will propagate faster and be more reliable.\todo{source}

In order to tell if a OGM has been received once or more than once it contains a sequence number, given by the originator of the OGM.
Each node re-broadcasts each received OGM at most once and only those received from the neighbour which has been identified as the currently best next hop (best ranking neighbour) towards the original initiator of the OGM.

This way the OGMs are flooded selectively through the network and inform the receiving nodes about other node's existence. 
A node X will learn about the existence of a node Y in the distance by receiving it's OGMs, when OGMs of node Y are rebroadcasted by its neighbours.

The algorithm then selects this neighbour as the currently best next hop to the originator of the message and configures its routing table respectively \cite{BATMAN}.

\subsection{Protocol comparison}
Below is a comparison table with the protocol discuses in this chapter.
The four different protocols is compared in what route metric they are using, if they are loop free, support of load balancing, how reliable they are, and the estimated throughput.
This table is to create an overview of the protocols and to support a later choice of protocol of implementation.

\begin{table}[!ht]
\caption{Protocol comparison}
\label{my-label}
\makebox[\linewidth]{
\begin{tabular}{|l|c|c|c|c|c|c|}
\hline
            & Ad hoc & Route metrics                                                                     & \begin{tabular}[c]{@{}c@{}}Loop \\ Free\end{tabular} & \begin{tabular}[c]{@{}c@{}}Load \\ balancing\end{tabular} & Reliability & Throughput                                                                    \\ \hline%cline{1-1}
\rowcolor[HTML]{EFEFEF} 
TDMA\cite{TDMATable}        & Yes    & \begin{tabular}[c]{@{}c@{}}Routes ensuring\\  guaranteed bandwidth\end{tabular} & Yes                                                  & Yes                                                       & High        & \begin{tabular}[c]{@{}c@{}}Decreases as more \\ nodes are added\end{tabular} \\
DSR\cite{ProtocolTable}\cite{DSR}         & Yes    & Source routing                                                                   & Yes                                                  & No                                                        & High        & \begin{tabular}[c]{@{}c@{}}Decreases as more\\  nodes are added\end{tabular} \\
\rowcolor[HTML]{EFEFEF} 
AOVD\cite{ProtocolTable}        & Yes    & \begin{tabular}[c]{@{}c@{}}Fastest \&\\ shortest path\end{tabular}               & Yes                                                  & No                                                        & High        & \begin{tabular}[c]{@{}c@{}}Poor for more\\  than 20 nodes\end{tabular}       \\
B.A.T.M.A.N\cite{BATMAN} & Yes    & \begin{tabular}[c]{@{}c@{}}Fastest \&\\ shortest path\end{tabular}               & Yes                                                  & No                                                        & High        & \begin{tabular}[c]{@{}c@{}}Good - scales\\ well with more nodes\end{tabular} \\ \hline
\end{tabular}}
\end{table}

\subsection{Flooding as a routing algorithm}
Flooding in networks, is a routing algorithm used to deliver data out though a system.
There exists two kind of flooding, uncontrolled flooding and controlled flooding.
Uncontrolled flooding is a kind of "brute force", where every node, in a network, keeps sending the same packets indefinitely out to all its neighbours within range, creating what is known as a broadcast storm.
The packet will eventually the packet will reach its destination, without knowing when\cite{flooding}.

Controlled flooding is modified algorithms to make it reliable, some of algorithms are knows as Sequence Number Controlled Flooding(SNCF) and Reverse Path Flooding(RPF).
With SNCF each node have its own address that is delivered with each data packet it transmits.
If a node receive a packet twice or more with the same address, the packet is discarded, so only one instants of the packet is working its way though the network.
In RPF, the node will forward a copy of a received packet, no matter the origin.
RPF except that the original transmitted packet will be copied by the right nodes and eventually end at the destination\cite{RPF}.

Advantaged of flooding, is that if there exists a route from the source to the destination, a packet will be delivered, and possible multiple times.
Since some flooding algorithms utilized every route in the network, it will also transfer packets though the shortest path\cite{flooding}.

Disadvantages can be that the bandwidth is wasted, since flooding is a very costly algorithm since it utilize every route.
Packets can easily\todo{fix} be duplicated in the network, future increasing bandwidth load.
Duplicate packet may loop in the system forever, if no prevention is made\cite{flooding}.

\todo{flooding: take 2}
\subsection{Flooding protocols}
Flooding in networks is a protocol used to deliver data throughout a network.
There are two kind of flooding: uncontrolled flooding and controlled flooding.\todo{source}

Uncontrolled flooding is the broadcasting of a packet, with a particular recipient, but without a particular route to reach the destination. Each node receiving the packet will repeat it to all of its neighbours, and they will repeat it again to theirs. This causes the packet to reach all nodes in the network, but it can cause communication loops. \cite{flooding}.

Controlled flooding is flooding protocols with filters or other measures to make it more reliable or efficient than the uncontrolled.\todo{source}
Examples of such algorithms are Sequence Number Controlled Flooding (SNCF) and Reverse Path Flooding (RPF).
With SNCF, each node have its own address that is delivered with each data packet it transmits.
If a node receive a packet multiple times with the same address\todo{what same address?}, the packet is discarded, so only one instance of the packet is transmitted.
In RPF, a node will forward a copy of a received packet, no matter the origin.
RPF except\todo{undtaget hvad?} that the original transmitted packet will be copied by the right\todo{what is right?} nodes and eventually end at the destination\cite{RPF}.

Advantages of flooding, is that if there exists a route from the source to the destination, a packet will be delivered, and up to multiple times, depending on the implemented protocol.
Since some flooding algorithms utilizes every route in the network, it will also transfer packets through the shortest path\cite{flooding}.

Disadvantages can be that the bandwidth is wasted, since flooding is a very costly algorithm since it utilize every route.\todo{Is bandwidth a problem? Isn't the problem battery life regarding to number of messages transmitted?}
Packets can easily\todo{fix} be duplicated in the network, further increasing bandwidth load.
Duplicate packet may loop in the system forever, if no prevention is made \cite{flooding}.
