\section{Golf course}
In this section, the physical and technical aspects of a golf course will be described. These factors are analyzed to be able to define requirements for the solution and to formulate a problem statement.

Firstly, the physical and geographical aspect will be examined. Thereafter, special elements of the task are studied to argue for choice of possible hardware for the solution. 

\subsection{Geography/topography}
This subsection will contain analysis and descriptions of golf courses' physical properties.

\textbf{Physical characteristics}\\
Golf courses have no strict geographical requirements, and each golf course is different, hence these characteristics are guidelines and conventions only, and not necessarily applicable to all golf courses. This section is mainly based on \cite{golfCourse}.

A golf course is typically covering a large area, and has up to 18 holes \cite{golfCourse}. As courses mainly are located outside, they are susceptible to weather and this requires possible devices placed on the golf course to be weather sealed, to avoid replacing them for example after a storm or heavy rain. Golf courses, as a problem domain, is considered static. This means that alterations to the original problem should not suddenly appear. % and there would be time for precautions or restructuring. 

The area of the golf course contains different types of flora in the regions. There can be open fields of grass, wood areas, grass or rock hills of varying sizes, sand traps as well as ponds or lakes. Some of these areas can create difficulties for communication between devices, either by affecting wireless signals by delay, dampening or reverberation, or by being obstacles for laying cables. 

\textbf{Game areas}\\
The different areas of the golf course are related to the parts of the game. The \textit{tee} is the start area. It usually has a level stance and short grass. The \textit{fairway} is the physically largest area and has longer grass. The \textit{rough} is the part surrounding the main playing area and can also be played on, but does not require the same amount of maintenance as the other areas, in accordance with the name. The \textit{green} is the destination area of a hole, and is closely mown and cared for. 

The areas are treated differently and have varied need of attention because of the grass' and soils properties. The green requires detailed maintenance due to the precision necessary for the golfers last strokes.

\textbf{Other characteristics}\\
Golf courses has people walking all over, and there is a chance for objects on it getting hit by golf balls, as stated in the interview. Therefore, the devices to be placed on a golf course should be tough, replaceable, repairable or be placed in such a way that they avoid damage. Golf courses are used frequently, but it is not highly critical for the whole course to always be operational or monitored, thus digging cables for sensors communication could be managed. The golf course should, according to the interviewee, not be dug up much, thus arguing for a wireless solution.

\subsection{Grass and soil properties}
A part of maintaining the grass is to consider the soils moisture \cite{golfGrass}. This can be measured by several methods, for example by visible indications or direct assessment by a sensor. Currently, in accordance with the interview with Kim, this is monitored by using visual aids and indications and not by sensors. 

Soil mostly consists of dirt and sand \todo{citationNeeded}, and has properties like density, moisture, acidity and water throughput. The grass should be watered not too little and not too much, and the level of watering and other care depends on the soils properties. The properties can vary due to weather and season, and accordingly require more or less attention in care and treatment. 
