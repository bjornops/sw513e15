\section{Golf course}
In this section, the physical and technical aspects of a golf course will be described. These factors are analyzed to be able to define requirements for the solution and to formulate a problem statement.

Firstly, the physical and geographical aspect will be examined. Thereafter, special elements of the task are studied to argue for choice of possible hardware for the solution. 

\subsection{Geography/topography}
This subsection will contain analysis and descriptions of golf courses' physical properties.

\textbf{Physical characteristics}
A golf course is typically covering a large area, and has up to 18 holes \cite{golfCourse}. 
As they mainly are located outside, they are susceptible to weather. This requires devices used on the golf course to be weather sealed, to avoid replacing them for example after a storm.
Golf courses are static by nature and changes to the area are organizationally applied. 

\textbf{Game areas}
The different areas of the golf course are related to the parts of the game. The \textit{tee} is the start area. It usually has a level stance and short grass. The \textit{fairway} is the physically largest area and has longer grass. The \textit{rough} is the part surrounding the main playing area and can also be played on, but does not require the same amount of maintenance as the other areas, in accordance with the name. The \textit{green} is the destination area of a hole, and has to be closely mown and cared for \todo{relevance - connect to soil quality and properties}. 

\cite{golfGrass}

\textbf{Other characteristics}
It has people walking all over, and there's a chance for objects on it getting hit by golf balls. 
They are used frequently, but it is not critical for the whole course to always be operational (digging cables for sensors). 



\subsection{Soil properties}
A part of the task of maintaining the soil, is to consider the soils moisture. This can be measured by several methods, for example by visible indications or direct assessment by a sensor. Currently, in accordance with the interview with Kim, this is monitored by using visual aids and indications. Soil mostly consists of dirt and sand.


\subsection{Watering - slightly OT?}
