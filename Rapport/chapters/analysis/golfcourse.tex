\section{Golf course}
In this section, the physical and technical aspects of a golf course will be described. These factors are analyzed to be able to define requirements for the solution and to formulate a problem statement.

Firstly, the physical and geographical aspect will be examined. Thereafter, special elements of the task are studied to give grounds for choice of possible hardware for the solution. 

\subsection{Geography/topography}
This subsection will contain well formulated descriptions and analysis of the following: A golf course is typically big, and has up to 18 holes. It has people walking all over, and there's a chance for objects on it getting hit by golf balls. They are mainly located outside and are accordingly susceptible to weather. They are used frequently, but it is not critical for the whole course to always be operational (digging cables for sensors). Golf courses are static by nature and changes to the region are organizationally applied. 

\subsection{Soil properties}
A part of the task of maintaining the soil, is to consider the soils moisture. This can be measured by several methods, for example by visible indications or direct assessment by a sensor. Currently, in accordance with the interview with Kim, this is monitored by using visual aids and indications. Soil mostly consists of dirt and sand.