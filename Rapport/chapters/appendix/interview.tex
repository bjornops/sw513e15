\chapter{Interview}
\textbf{Hvilke ting holder i øje med, i henhold til vedligeholdelse af golfbanen?}\\
Hvordan beplantningen groer, træer og græs.

\textbf{Er det nogle arbejdsopgaver i kunne se automatiseret?}\\
Set fra golf banens ledelse så ville automatiserede maskiner være gode da de så ville kunne spare greenkeeperen.
Ville dog ikke være til vores fordel som greenkeepere

Med et middel der kunne gøre vi kunne se jordens tilstand med surhed og vandmængde, ville vi lettere kunne tilføre hvad der mangler af, gødning og vand

\textbf{Ville en automatisering resultere i et besparelse eller anden fordel for jer?}\\
Automatiske klippere, hvilket så er lavet 


\textbf{Hvor vigtigt er det for jer at undersøge jordfugtigheden?}\\
Vigtigt på greens, i forhold til om de tørrer ud. Sker ikke så ofte, men er relevant.
Kan også være relevant på teested.

\textbf{Hvordan gør i det nu?}\\
Når hullerne flyttes undersøges dette.

\textbf{Hvor ofte undersøger i det nu?}\\
Mest om sommeren, da der kan udtørres. Planlægger det efter vejret, her kan produktet være til hjælp for at spare det arbejde.

\textbf{Ville det være attraktivt med et system der automatisk undersøger det?}\\
Ja det vil det. Der kan være forskel på forskellige steder på banen, da det kan regne nogle steder uden at det gør andre.

\textbf{Kender i allerede til systemer til dette formål?}\\
RainBird har et lignende system. Disse har også leveret styreprogram til vandingsanlæg til Aalborg Golfklub.

\textbf{Vil det give mening at sende informationerne trådløst i stedet for gennem kabler?}\\
Ja, så slipper vi for at grave dem over hver gang vi skal reparere noget på banen. Kan give problemer med ledninger i forvejen og vandingsanlæg. Især problemer med rødder på træer.

\textbf{Hvor på banen skulle sensorer placeres? (Og hvordan? dybde? jord/sand?)}\\
Mellem 10-15 cm. nede, da der skal være rødder på greenen. Der er ca. 30cm. sand på greenen, som vandet hurtigt ryger igennem. Rødder er ca. 10-15cm. nede, der hvor man gerne vil have det er fugtigt.
En green kan være ret stor, så det kan være nødvendigt at placere flere på en green. En green kan være $500m^2$.

Teesteder er ikke så vigtigt.

\textbf{Nogle potentielle problemstillinger ved et sådant system?}\\
Når jorden prikkes kan de rammes og evt. gå i stykker. Kan undgås ved at grave lidt ned, Kim mener at de prikker 10cm. ned. 
At finde dem efter de er gravet ned. Kan hurtigt blive et problem. Batteritid er vigtigt, kan blive et problem.

\textbf{Nogle idéer til anvendelse eller forbedringer/udvidelser til systemet?}\\
pH-meter indbygget vil være smart i forhold til at udregne brugen af gødning og kalk for at få den ønskede pH værdi (Græs gror bedst ved pH 5.6, varierer efter græssort, men under 7)
Hvorvidt jorden er porøs eller ikke porøs, da dette skal bruges til at finde ud af om jorden skal prikkes.
Jord har godt af at blive prikket, da for tæt jord holder på vandet og dette kan stoppe alt vækst og materiale.
