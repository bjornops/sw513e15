\chapter{Interview} \label{cha:interviewKim}
Two interviews were performed with Kim. The first one to gain an understanding of which problems greenkeepers might have, that could be solved with an embedded system. The second one to verify the usefulness of the idea to monitor properties of the golf course.

\section*{Initierende interview}
\textbf{Hvilke ting holder i øje med, i henhold til vedligeholdelse af golfbanen?}\\
Hvordan beplantningen som træer og græs gror.

\textbf{Er det nogle arbejdsopgaver i kunne se automatiseret?}\\
Automatiserede klippere, hvilket dog allerede er lavet.

Med et middel der kunne gøre, at vi kunne se jordens tilstand med surhed og vandmængde, ville vi lettere kunne tilføre hvad der mangler af gødning og vand.

\textbf{Ville en automatisering resultere i et besparelse eller anden fordel for jer?}\\
Set fra golfbanens ledelse så ville automatiserede maskiner være gode, da de ville kunne spare greenkeeperen.
Dette ville dog ikke være til vores fordel som greenkeepere

\section*{Uddybende interview}

\textbf{Hvor vigtigt er det for jer at undersøge jordfugtigheden?}\\
Det er vigtigt på greens i forhold til, om de tørrer ud. Dette sker ikke så ofte, men er stadig relevant.
Det kan også være relevant på teesteder.

\textbf{Hvordan gør i det nu?}\\
Når hullerne flyttes undersøges dette.

\textbf{Hvor ofte undersøger i det nu?}\\
Mest om sommeren, da der oftest sker udtørring. Jeg planlægger det efter vejret, og her kan produktet være til hjælp for at spare dette arbejde.

\textbf{Ville det være attraktivt med et system der automatisk undersøger det?}\\
Ja det vil det. Der kan tilmed være forskel på forskellige steder på banen, da det kan regne nogle steder uden at det gør andre.

\textbf{Kender i allerede til systemer til dette formål? (Her refereres til undersøgelse af jordfugtigheden, som tidligere er nævnt)}\\
RainBird har et lignende system. Disse har også leveret styreprogram til vandingsanlæget til Aalborg Golfklub.

\textbf{Vil det give mening at sende informationerne trådløst i stedet for gennem kabler?}\\
Ja, så slipper vi for at grave dem over hver gang vi skal reparere noget på banen. Dette kan give problemer med allerede eksisterende ledninger og vandingsanlæg. Især også med rødder på træer.

\textbf{Hvor på banen skulle sensorer placeres? (Og hvordan? dybde? jord/sand?)}\\
Mellem 10-15 cm. nede, da der skal være rødder på greenen. Der er ca. 30cm. sand på greenen, som vandet hurtigt ryger igennem. Rødder er ca. 10-15cm. nede, der hvor man gerne vil have det er fugtigt.
En green kan være ret stor, så det kan være nødvendigt at placere flere på en green. En green kan sagtens være $500m^2$.

Teesteder er ikke så vigtigt.

\textbf{Ser du nogen potentielle problemstillinger ved et sådant system?}\\
Når jorden prikkes kan de rammes og evt. gå i stykker. Kan undgås ved at grave lidt ned. Jeg mener at de prikker ca. 10cm. ned. 
At finde dem efter de er gravet ned kan hurtigt blive et problem. Batteritid er vigtigt og kan også blive et problem.

\textbf{Nogle idéer til anvendelse eller forbedringer/udvidelser til systemet?}\\
pH-meter indbygget vil være smart i forhold til at udregne brugen af gødning og kalk for at få den ønskede pH værdi (Græs gror bedst ved pH 5.6, men varierer efter græssort, men som regel under 7)
Hvorvidt jorden er porøs eller ikke porøs, da dette skal bruges til at finde ud af om jorden skal prikkes.
Jord har godt af at blive prikket, da for tæt jord holder på vandet og dette kan stoppe alt vækst og materiale.
