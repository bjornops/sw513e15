\chapter{Checksum test code} \label{cha:checksumtest}
\begin{lstlisting}[language=C]
// Create packet
Packet checkPacket(Data, 1, 2, 3, 4, 5, 6);

if(checkPacket.checksumMatches())
{
    std::cout << "Packet is valid after initializing" << std::endl;
}

// Change values and check if packet is still correct
// PacketType
checkPacket.packetType = DataRequest;
if(!checkPacket.checksumMatches())
{
    std::cout << "Packet is incorrect after changing packetType" << std::endl;
}
checkPacket.packetType = Data;

// Addresser
checkPacket.addresser = 2;
if(!checkPacket.checksumMatches())
{
    std::cout << "Packet is incorrect after changing addresser" << std::endl;
}
checkPacket.addresser = 1;

// Addressee
checkPacket.addressee = 1;
if(!checkPacket.checksumMatches())
{
    std::cout << "Packet is incorrect after changing addressee" << std::endl;
}
checkPacket.addressee = 2;

// Origin
checkPacket.origin = 1;
if(!checkPacket.checksumMatches())
{
    std::cout << "Packet is incorrect after changing origin" << std::endl;
}
checkPacket.origin = 3;

// Sensor 1
checkPacket.sensor1 = 1;
if(!checkPacket.checksumMatches())
{
    std::cout << "Packet is incorrect after changing sensor1" << std::endl;
}
checkPacket.sensor1 = 4;

// Sensor 2
checkPacket.sensor2 = 1;
if(!checkPacket.checksumMatches())
{
    std::cout << "Packet is incorrect after changing sensor2" << std::endl;
}
checkPacket.sensor2 = 5;

// Sensor 3
checkPacket.sensor3 = 1;
if(!checkPacket.checksumMatches())
{
    std::cout << "Packet is incorrect after changing sensor3" << std::endl;
}
checkPacket.sensor3 = 6;

// Test again with default values
if(checkPacket.checksumMatches())
{
    std::cout << "Packet is correct with default values" << std::endl;
}
\end{lstlisting}
