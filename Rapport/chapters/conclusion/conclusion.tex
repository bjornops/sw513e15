\section{Conclusion}


In the analysis it was found that a system, which could monitor the moisture in the soil, was a solution that would help reducing the amount of work required in keeping a golf course.

Next the group analysed different technologies and solutions for the problems found. This led to a final design where technologies for the solution were chosen, either as 'the best' or as 'best tradeoff'.

Finally the solution could be implemented using the chosen design and developing principles. For example pair programming were used a lot to achieve higher quality code, and wider understanding of the code. Together with the implementation, testing were performed frequently to verify system behavior. Also, hardware components were tested to verify reliability.

The problem statement from the analysis were as follows:
\textit{How can a sensor network and a corresponding protocol be designed for a golf course, so that data can be relayed throughout the network, enabling an endpoint device to receive the information without being within range of all sensors in the network?}

The outcome of this project is a network of devices that can transmit data through each other, to a main node wirelessly, by following a protocol defined for this purpose. The main node does not have to be in range of all nodes to receive their data.

The data transmitted is ground moisture readings, and meta packets to relay network relevant data needed to use for doing so. Furthermore the main node contains of a web interface where a user can request data from the network and view the received data.
