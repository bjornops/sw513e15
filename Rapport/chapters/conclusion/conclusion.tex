\section{Conclusion}
In the analysis it was found that a system, which could monitor the moisture in the soil was a solution that would optimize the amount of work required in keeping a golf course.

Furthermore it was found, that the system would be preferred as wireless, avoiding the need of digging down cables, and also the ability for the sensors to be moved freely into new areas without any restructuring.

Later in the analysis, some problem aspects were found to avoid any unseen challenges that would ruin the whole solution. Here, Power consumption, Wireless range and security, were considered. By means of that, it was chosen to let the devices communicate through each other to reduce power consumption and potentially extend wireless range.

Next the group analysed and found many different technologies and solutions for the problems found. This led to a final design where a solution were chosen, either as 'the best' or as 'best tradeoff'.

Finally the solution could be implemented using the chosen design and developing principles. For example pair programming were used a lot to ensure good quality code, and wider understanding of the code. Together with the implementation, testing were performed frequently to verify system behavior. Also, hardware components were tested to verify reliability.

The problem statement from the analysis were as follows:
\textit{How can a sensor network and a protocol be designed, so that data can be relayed throughout the network, enabling an endpoint device to receive the information without being within range of all sensors in the network?}

The outcome of this project is a network of devices that can transmit data through each other, to a main node wirelessly. The main node does not have to be in range of all nodes to receive their data.

The main data transmitted is ground moisture readings, but also network commands are distributed throughout the network. Furthermore the mainnode consists of an interface where a user can request data from the network and view the received data.