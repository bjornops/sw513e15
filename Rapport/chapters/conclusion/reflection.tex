\section{Reflection}
Generally, the development process has been working fine and gave the group some powerful tools in the project. The choice of developing a prioritized requirements specification made the group able to utilize the whole project period, with 'nice to have' features if the main solution were to be done too early, thus having an iterative implementation process.

In the design and implementation phase there were several iterations because subproblems were found several times. These problems had to be described, analysed and handled in the solution before proceeding. Though, this were to be expected because the problem domain were new to the group, and therefore hard to cover all aspects.

Pair programming worked out well in the group, and led to greater understanding of the code. Since atleast two members in the group were able to explain the thoughts behind some code, information sharing among the whole group could be done in a simpler manner. Furthermore, pair programming led to better quality solution because a pair could discuss potential implementations before actually choosing one.

The planning tool, Trello, used in the project did cause problems. Most members in the group forgot to update the scrum board, and therefore it was not possible to track the process and state of all tasks. This led to bad overview, which is opposite of the plan with Trello. Luckily the group did not run into any conflicts or 'double work' during the project.
