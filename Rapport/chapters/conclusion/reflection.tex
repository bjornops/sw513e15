\section{Reflection}
The working process has led the project group some powerful tools in the project. The choice of developing a prioritized requirements specification made the group able to utilize the whole project period with 'nice to have' features if the main requirements were to be implemented too early, and having an iterative implementation process.

In the design and implementation phase there were several iterations because subproblems were found several times. These problems had to be described, analysed and handled in the solution before proceeding. Though, this was to be expected due to none of the group members having developed a protocol before.

Pair programming worked out well in the group, and led to greater understanding of the code. Since at least two members in the group were able to explain the thoughts behind the code, information sharing among the whole group could be done in a simpler manner. Furthermore, pair programming led to a higher quality solution because a pair could discuss potential implementations before actually choosing one.

The planning tool, Trello, used in the project did cause problems, as members in the group forgot to update Trello. Therefore it was harder to track the progress and state of all tasks. Still, the group did not run into any conflicts or 'double work' during the project.

The solution fulfills the requirements, but some aspects from the analysis were completely left out. Explicit power saving features were not implemented anywhere, which resulted in a system that might run out of power. Some functions have been implemented with power saving in mind, eg. that the system operates on demand instead of automatically monitoring, or that a node will eventually stop trying to broadcast its data.

For the solution to be finally verified as functioning, it would be required to test the solution on an actual golf course with a realistic number of nodes and some actual users. This is the only way to verify that the system actually works in a real situation.