This chapter contains documentation about the implementation of the protocol designed in Chapter \ref{cha:protocolDesign}. 

First, the classes used in the implementation are shown and explained. Then important parts of the sourcecode are explained.
 

\section{General}
The implementation is written in C++. The development of the solution is done in two man teams, using pair programming. Git is used for version control.

\section{Classes}
The solution contains two different running applications. The main node and the sensor nodes. There are two sets of code due to the different requirements of the nodes. For example, the main node does not have a sensor, so it should not be able to handle a sensor, and therefore does not contain the classes used for sensors.

This section contains the classes contained in each application.

\subsubsection*{Main node}
The main node is a Raspberry Pi device, running the Raspbian operating system. The classes in the main node application is seen on Figure \ref{fig:mainnodeClass}.
The main node pairs nodes, send requests, and handles receiving and handling data.

\begin{figure}[h!]
\centering
\includegraphics[width=0.65\textwidth]{chapters/implementation/figures/mainnodeClass.png}
\caption{Classes used in the main node.}
\label{fig:mainnodeClass}
\end{figure}



\subsubsection*{Sensor nodes}
The platform on the sensor nodes are comprised of Arduino Uno or Mega. The difference in the two platforms are the pins used with the sensors and radio modules. Besides that, the code on the platforms are the same. The sensor nodes handles receiving and sending requests, reading and sending sensor data, and relaying data through nodes until they arrive at the main node. 

The classes used in the nodes are seen on Figure \ref{fig:nodeClass}.
\begin{figure}[h!]
\centering
\includegraphics[width=0.85\textwidth]{chapters/implementation/figures/nodeClass.png}
\caption{Classes used in nodes.}
\label{fig:nodeClass}
\end{figure}


\subsubsection*{Class descriptions}
This section contains explanations about the classes used in the nodes.

\begin{description}
\item[Node] \hfill \\
The \textit{Node} class is the entrypoint in the application.
It Contains one or more \textit{iSensor}s, and a \textit{iRadio} object that are instantiated when the code is run.

The \textit{Node} class determines what action to take when a packet is received. This happens in the method \textit{void determineAction(char* packet)}.
If data is received from another sensor node that should be relayed, the \textit{Node} begins sending data until an acknowledgement is received. The methods on the \textit{iRadio} object is used for listening and sending.
If data is needed for a packet, for example when a request is received, the \textit{Node} class fetches data from the sensor, and creates the packet that needs to be sent, and makes sure that the data is sent.

\item[iRadio and iSensor] \hfill \\
The \textit{iSensor} and \textit{iRadio} interfaces are used to create a certain way that all sensors and radio module classes should work. This means that replacing or adding a new sensor or using another radio module is not a problem, as the rest of the application knows that the radio is able to send and receive data, and the sensor is able to return a value. 
It is not necessary to know how it is done, only that it is possible. These interfaces are used to set a standard on how to communicate with them, whether it is a moisture sensor or a pH sensor.

Every type of radio or sensor added to the node must inherit from these interfaces.


\textit{iSensor} only contains the method \textit{int read()}, that returns the value of the sensor.

\textit{iRadio} contains the methods:
\begin{description}
\item[void broadcast(char *packet)] sends a packet
\item[char *beginListening()] begins listening
\item[char *listenFor(int ms)] begins listening, but stops listening after 'ms' ms
\end{description}

\item[Sensor] \hfill \\
The \textit{Sensor} class can be used for multiple sensors, as long as the sensor class inherits from \textit{iSensor}. This class' job is to read data from a sensor on the node, handle this data, and return it to the class that requested this data.

In the solution, as the moisture sensor is used, the sensor class is called \textit{MoistureSensor}. This class can only be found on the sensor nodes, and not on the main node.


\item[Radio] \hfill \\
The \textit{Radio} class handles the radio communication. Types of this class should inherit from the \textit{iRadio} interface. The \textit{Radio} class returns an array of char pointers when one is received. It is also able to send packets given from other classes.

\item[Packet] \hfill \\
The class \textit{Packet} contains the ability to parse a packet from a char array to a \textit{Packet} object, along with the properties required to further handle the packet. This includes the type, and possibly sensor values, from/to values or an identifier from the main node.

Instances of this class is created and used in the \textit{Node} class.

This class contains the public methods:
\begin{description}
\item[Packet(char *input)] Create a packet from a char array. Used when radio receives a packet.
\item[Packet(PacketType packetTypeInput, uint16\_t addresserInput, uint16\_t addresseeInput, uint16\_t originInput, uint16\_t sensor1Input, uint16\_t sensor2Input, uint16\_t sensor3Input)] Creates a new packet with data as noted in parameters.
\item[char *encode()] Returns a char array representation of the packet. Used for broadcasting from the radio.
\end{description}

\end{description}


\section{Code}
This section contains explanations for some of the important parts of the code.
\textbf{Moar.}


\subsection{Packet}
The \textit{Packet} class contains the data from a packet received from the radio module. It can be instantiated with a string, or with every part of the packet as a parameter, using the constructors:
\begin{lstlisting}
Packet::Packet(PacketType packetTypeInput, uint16_t addresserInput, uint16_t addresseeInput, uint16_t originInput, uint16_t sensor1Input,
	uint16_t sensor2Input, uint16_t sensor3Input)
	
Packet::Packet(char *input)
\end{lstlisting}


This class is passed around in \textit{Node}, where it is used to determine actions based on the type, or being relayed with some new data. 
The class itself contains the variables:
\begin{lstlisting}
PacketType packetType;
uint16_t addresser;
uint16_t addressee;
uint16_t origin;
uint16_t sensor1;
uint16_t sensor2;
uint16_t sensor3;
uint16_t checksum;
\end{lstlisting}
These values cover everything in a packet; the type, the sender, the receiver, the origin of the packet, values and a checksum.
Every packet has this format, and all values are filled out. With packets that might not require all these values, they will simply be zero. For example when sending a request, the sensor values and addressee will be zero.

\begin{table}[]
\centering
\begin{tabular}{|l|c|l|c|l|c|l|c|l|}
\hline
\textbf{Datatype} & \multicolumn{2}{c|}{Packettype}      & \multicolumn{2}{c|}{uint16\_t}       & \multicolumn{2}{c|}{uint16\_t}       & \multicolumn{2}{c|}{uint16\_t}       \\ \hline
\textbf{Name}     & \multicolumn{2}{c|}{packetType}      & \multicolumn{2}{c|}{addresser}       & \multicolumn{2}{c|}{addressee}       & \multicolumn{2}{c|}{origin}          \\ \hline
\textbf{Memory}   & \multicolumn{1}{l|}{8 bits} & 8 bits & \multicolumn{1}{l|}{8 bits} & 8 bits & \multicolumn{1}{l|}{8 bits} & 8 bits & \multicolumn{1}{l|}{8 bits} & 8 bits \\ \hline
\end{tabular}
\caption{Dataoverview of first half of a packet. The packet can be directly converted to 16 characters.}
\label{tab:packetTableFirst}
\end{table}

\begin{table}[]
\centering
\begin{tabular}{|l|c|l|c|l|c|l|c|l|}
\hline
\textbf{Datatype} & \multicolumn{2}{c|}{uint16\_t}       & \multicolumn{2}{c|}{uint16\_t}       & \multicolumn{2}{l|}{uint16\_t} & \multicolumn{2}{c|}{uint16\_t}       \\ \hline
\textbf{Name}     & \multicolumn{2}{c|}{sensor1}         & \multicolumn{2}{c|}{sensor2}         & \multicolumn{2}{c|}{sensor3}   & \multicolumn{2}{c|}{checksum}        \\ \hline
\textbf{Memory}   & \multicolumn{1}{l|}{8 bits} & 8 bits & \multicolumn{1}{l|}{8 bits} & 8 bits & 8 bits         & 8 bits        & \multicolumn{1}{l|}{8 bits} & 8 bits \\ \hline
\end{tabular}
\caption{Dataoverview of second half of a packet.}
\label{tab:packetTableSecond}
\end{table}

The datatype of the members of \textit{Packet} are all \textit{uint16\_t}. This is to ensure that the size of the class does not vary on different architectures. The \textit{uint16\_t} is always of size 2 bytes. The same size as two characters. Together, all members in the class gives a total size of 16 bytes. The final architecture of a packet can be seen on \ref{tab:packetTableFirst} and \ref{tab:packetTableSecond}.

Only half of the space available in a transmission are occupied as the radio module used in this project supports transmissions of 32 byte at a time. As the solution is designed modular, and the radio module can be replaced with a potentially 'smaller' module in terms of transmission size, the solution would still support this 'smaller' module. Furthermore, a packetsize of 16 bytes is enough to contain the different data collected from the nodes.

\textit{PacketType} is defined as follows:
\begin{lstlisting}
enum PacketType : uint16_t {
    Acknowledgement,
    Request,
    Data,
    PairRequest,
    PairRequestAcknowledgement,
    ClearSignal
};
\end{lstlisting}


When a \textit{Packet} is instantiated using the constructor with the \textit{char *input} as parameter, these values are found using the function \textit{memcpy}, in the \textit{decode} function:
\begin{lstlisting}
void Packet::decode(char *input)
{
    memcpy(this, input, sizeof(Packet));
}
\end{lstlisting}
This copies the memory of the \textit{char *input} into the memory location where the current \textit{Packet} object is located. This means that the contents of the packet is set to the content of \textit{input}.

This is possible as every part of the packet has a known, pre-determined size. \textit{uint16\_t} always has the same size, not dependant on the architecture running the code. When a packet is encoded, using the \textit{char *encode()} function, the returned character array is built in the same way, by setting the contents of the array to the contents of the package.

\begin{lstlisting}[language=C]
char *Packet::encode()
{
    char *returnstring;
    returnstring = (char*)malloc(sizeof(Packet));

    memcpy(returnstring, this, sizeof(Packet));

    return returnstring;
}
\end{lstlisting}
The \textit{encode} function allocates the size necessary for a \textit{Packet}, but as a \textit{char *} and uses \textit{memcpy} to set the contents of the array to the contents of the current \textit{Packet} and then returning the array. The char array can be seen as a format of the packet object, that can be directly transmitted by the radio module without further operations to the array.

\subsubsection{Checksum}
\textit{Packet} objects contains a checksum, used to verify the content's integrity. \todo{Bjørn indsæt CRC implementering her.} The checksum uses two functions: \texttt{crcInit()} and \texttt{crcCompute(unsigned char *message, unsigned int nBytes)}. The checksum implementation is based on the description in \textit{Programming Embedded Systems in C and C++} \cite{crcCode}.

The \texttt{crcInit()} should be run first to generate a table of remainders to expedite the computation of the checksum. The computation is accelerated because the remainder can be calculated byte-wise instead of bit-wise, by the cost of $256 * 2 \text{ bytes} = 512$ bytes. 

\begin{lstlisting}[language=C]
void Node::crcInit()
{
    unsigned short remainder; // 2 byte remainder
    unsigned short dividend; // What are you?
    int bit; // bit counter

    for(dividend = 0; dividend < 256; dividend++) //foreach value of 2 bytes/8 bits
    { 
        remainder = dividend << (WIDTH - 8);//

        for(bit = 0; bit < 8; bit++)
        {
            if(remainder & TOPBIT) // MSB = 1 => divide by POLYNOMIAL
            { 
                remainder = (remainder << 1) ^ POLYNOMIAL; //scooch and divide
            }
            else
            {
                remainder = remainder << 1;//scooch and do nothing
            }
        }
        crcTable[dividend] = remainder;//save current crc value in crcTable
    }
}
\end{lstlisting}

