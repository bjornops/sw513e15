\section{RF24 testing}
The following section contains the tests performed on the RF24 radio module. These tests were performed to determine how 

\subsection{RF24 reliability}
The first test performed on the RF24 is a reliability test. This test is performed using one sender and one receiver device. The objective of the test is to determine the number of packets lost from one device to another. 
The code used in this test can be found in Appendix \ref{cha:reliabilityCode}.


The sender device sends a number of packets before stopping, e.g. 10000 or more, using a delay between each package, though the delay can be removed. For every packet sent, the sender increments a counter value.

The receiver accepts packets, and for every packet received it increments a counter value. This value is printed to the console, to allow the tester to track the number of packets received.

Multiple delay values were tested, including no delay used. During these tests, only one packet did not reach the receiver, when using no delay at all. This could be due to a number of things, but most likely the receiver did not print the value to the console due to speed or the sender started before the receiver.

\textbf{Evt. tabel med fem gennemkørsler der viser resultater?} \\

The conclusion to the reliability test is that the RF24 is more than reliable enough for the requirements of this project.


\subsection{RF24 transreciever analysis}
To confirm the usability of the RF24 transreciever, labeled as RF24 from this point, a short analysis with tests will be performed.

\subsection{Range}
A basic test to test the range is useful to ensure that RF24 is usable in a practical application with a wide distance between units.

%\subsection{Testing method:}
Testing is done by having two units set as transmitter and reciever units respectively, having the transmitter transmit a single signal repeatedly while increasing the distance between the units. When the reciever no longer recieves the signal, the distance can then be measured.

CODE GOES HERE!

%\subsection{Results:}
\begin{table}[!h]
\begin{tabular}{|l|l|l|l|} \hline
	\diaghead{\theadfont Diag ColumnmnHead II} {Reciever}{Transmitter}
			 	& Unit 1 	& Unit 2 	& Unit 3 	\\\hline
	Unit 1  	& - 		& ? 		& ? 		\\\hline
	Unit 2  	& ? 		& - 		& ? 		\\\hline
	Unit 3  	& ? 		& ? 		& - 		\\\hline
\end{tabular}
\end{table}

\subsection{Bit error rate:}
what is?

%\subsection{Testing method}:
how? code, setup etc.

%\subsection{Results:}
\begin{table}[!h]
\begin{tabular}{|l|l|l|l|} \hline
	Datasheet 	& Unit 1 	& Unit 2 	& Unit 3 	\\\hline
	0,0\% 		& ? 		& ? 		& ? 		\\\hline
\end{tabular}
\end{table}