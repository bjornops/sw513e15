\section{RF24 transreciever analysis}
To confirm the usability of the RF24 transreciever, labeled as RF24 from this point, a short analysis with tests will be performed.

\subsection{Range}
A basic test to test the range is useful to ensure that RF24 is usable in a practical application with a wide distance between units.

%\subsection{Testing method:}
Testing is done by having two units set as transmitter and reciever units respectively, having the transmitter transmit a single signal repeatedly while increasing the distance between the units. When the reciever no longer recieves the signal, the distance can then be measured.

CODE GOES HERE!

%\subsection{Results:}
\begin{table}[!h]
\begin{tabular}{|l|l|l|l|} \hline
	\diaghead{\theadfont Diag ColumnmnHead II} {Reciever}{Transmitter}
			 	& Unit 1 	& Unit 2 	& Unit 3 	\\\hline
	Unit 1  	& - 		& ? 		& ? 		\\\hline
	Unit 2  	& ? 		& - 		& ? 		\\\hline
	Unit 3  	& ? 		& ? 		& - 		\\\hline
\end{tabular}
\end{table}

\subsection{Bit error rate:}
what is?

%\subsection{Testing method}:
how? code, setup etc.

%\subsection{Results:}
\begin{table}[!h]
\begin{tabular}{|l|l|l|l|} \hline
	Datasheet 	& Unit 1 	& Unit 2 	& Unit 3 	\\\hline
	0,0\% 		& ? 		& ? 		& ? 		\\\hline
\end{tabular}
\end{table}